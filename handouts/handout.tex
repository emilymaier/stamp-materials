 \documentclass[letterpaper]{tufte-book}
\usepackage{booktabs}
\usepackage{tabularx}
\usepackage{longtable} 
\usepackage{lscape}
\usepackage{colortbl}
\usepackage{graphicx}
\graphicspath{{../images/}}

\geometry{
  left=.5in,
  right=.5in,
  top=.5in,
  bottom=.5in
}

%\title{Creating a Hierarchical Control Model}

\begin{document}

\setlength{\parindent}{0em}
\setlength{\parskip}{.75em}

\section{Creating a Hierarchical Control Model}

\newthought{Definition}

A \textbf{hierarchical control model} is a representation of the system as a combination of control loops depicting the roles and responsibilities of different human and technical components and the relationships (in terms of authority and accountability) that they have to one another.

It consists of:
\begin{compactitem}
\setlength{\itemsep}{0pt}
\setlength{\parskip}{.25em}
\item a \emph{graphical representation} of the relationships components have to one another
%
%Note: in the sense of "nodes and edges" as well as "visual depiction"
\item \emph{annotations} capturing certain information about the system parts depicted
\end{compactitem}

\begin{center}
\includegraphics[width=4cm, height=4cm]{generic_model}
\end{center}

\newthought{Visual Conventions}

\begin{compactitem}
 \setlength{\itemsep}{0pt}
\setlength{\parskip}{.25em}
 \item \textsc{boxes}
 
 Each box can represent an abstract process, a human organization or group, an individual human operator, or a technical component.
 \item \textsc{nesting}
 
 Boxes inside others may be used to indicate that the interior elements are sub-processes of the exterior elements.
 \item \textsc{vertical position} 
 
 Items higher on the page exert authority over items lower on the page. 
\item \textsc{arrows} connecting boxes
\begin{compactitem}
		 \setlength{\itemsep}{0pt}
		\setlength{\parskip}{.25em}
		\item down: \emph{control actions} or exerting \emph{authority}
		\item up: \emph{feedback} via sensors, or \emph{accountability}
		\item horizontal: coordination and handoffs between \hbox{processes} or peers.
	\end{compactitem}
 \end{compactitem}
 

\textsc{Annotations}
\begin{compactitem}
\setlength{\itemsep}{0pt}
\setlength{\parskip}{.25em}
\item \textsc{goals} 

Specific target outcomes to guide the controlled process towards
\item \textsc{internal process model / mental model}

This includes variables representing the state of the controlled process as perceived by the controller. \emph{Note}: May differ from actual state.
\item \textsc{control algorithm / decision process}

How does the controller choose what actions to take and when?
We do not need to write out the whole algorithm here.
\item \textsc{Available control actions}

We can write a list of the actions and depict them graphically with arrows to the controlled process(es).
We can depict the \emph{actuators} that carry out the actions by noting them along the arrows.
\end{compactitem}

\begin{compactitem}
\item \textsc{Input and instructions} directed to the controller to set goals, etc.
\item \textsc{actuators} to carry out the actions the controller specifies (depicted on arrow edges)
\item \textsc{sensors}

Arrows back from the controlled process to the controller depict \emph{feedback via sensors}, updating the internal process model.
\item \textsc{system being controlled} --- Represented by another box, which may itself be a controller.
\end{compactitem}  

\newthought{Strategic Approaches} 
\begin{compactitem}
\setlength{\itemsep}{0pt}
\setlength{\parskip}{.25em}
\item Start simple, perhaps with only 3 rectangles.
\item Multiple diagrams at different levels of abstraction may be useful.
\end{compactitem}  

\section{Hierarchical Control Model Starting Point}

\newthought{Starting Questions}

\begin{itemize}
\setlength{\itemsep}{0pt}
\setlength{\parskip}{.25em}
\item What are some of the participants? (names of groups, components, etc.)
\item What responsibilities are present here? Whose are they?
\item What actions are available? Whose are they?
\item What decisions are being made and who or what is responsible for making them?
\item What information do components use to make those decisions, and how do they get it?
\end{itemize}

\newthought{Visual Conventions}

\begin{itemize}
\setlength{\itemsep}{0pt}
\setlength{\parskip}{.25em}
\item Items higher on the page exercise authority over the things lower on the page.
\item Arrows pointing down represent \emph{control actions} or exerting \emph{authority}.
\item Arrows pointing up represent \emph{feedback} via sensors, also describable as \emph{accountability}.
\item Horizontal lines represent coordination and handoffs.
\item Boxes may be nested, representing sub-processes.
\end{itemize}

\begin{center}
\includegraphics[width=10cm]{generic_model.png}
\end{center}

\newthought{Relationship to other concepts}

%The \textbf{actions} we list in our model are ones we can analyze to identify \textbf{unsafe control actions}.

When we identify \textbf{causal scenarios} that can lead \textbf{unsafe control actions} to occur, we seek flaws in the control loop the action is part of, using our model to show us what actuators and feedback are relevant.

\pagebreak

\newthought{Thermostat Example}

\input{examples/thermostat/model}

Note: The ACTUAL temperature may differ from MEASURED (e.g. ACTUAL might be 68 degrees F while MEASURED is 69 degrees F).

 \newthought{Thermostat Diagram}

\begin{center}
\includegraphics[width=8cm]{thermostat_model}
\end{center}

\section{Identifying Losses}

\newthought{Definition}

Accident: An undesired or unplanned event that results in a \textbf{loss}, including loss of human life or human injury, property damage, environmental pollution, mission loss, etc. [\emph{Engineering A Safer World} p 181]

\textbf{Losses} are the outcomes we want to prevent.

Our task is to \textbf{write a list} of them, staying broadly general and covering all the areas of concern for our safety analysis to address.
\newthought{Thermostat Example}

\input{examples/thermostat/losses}

\newthought{Desired Qualities} 
\begin{itemize}
\setlength{\itemsep}{0pt}
\setlength{\parskip}{.25em}
\item Concise --- We want a relatively short list (<20); these are our priorities.
\item General --- We don't want to prematurely narrow our focus. 

Example: "Someone is injured" may be a more useful loss statement than "Someone is injured by hot equipment" because, overall, we want to prevent \emph{any} injury.
\item Good coverage --- For any accident we can think up, we want it to be described by at least one of the losses on this list.
\item Non-redundant --- Overlap between losses is ok, but if one loss is entirely a subset of another, perhaps consider consolidating them.
\item Relevant --- They should be problems we actually consider important to prevent for our system. 

Example: "Civil war breaks out" is not a loss relevant to our thermostat example (except jokingly, or if there's an allegory about climate change).
\end{itemize}  
 
\newthought{Strategic Approaches} 

Ask "what is \emph{unacceptable}?" vs. "what is \emph{risky but tolerable}?" to distinguish from hazards.

\section{Generic Losses}

\newthought{Definition}

\textbf{Losses} are the outcomes we want to prevent.

\newthought{What this list is}

Col. Bill Young, a colleague of Nancy Leveson's, proposes considering mission loss in some of his talks.
Much of this list grew out of conversations in Akamai's InfoSec Safety Team about "life, limb, liberty, loot..." and gradually expanded. This list may not be complete, and you may wish to subdivide these items differently, but it can serve as an inspirational starting point.

\newthought{Generic Losses to start with} 
\begin{enumerate}
\setlength{\itemsep}{0pt}
\setlength{\parskip}{.25em}

\item \textsc{mission loss} - Failure to meet critical system goals and "minimum viable product" requirements
\item \textsc{life and limb} - Accidents resulting in injury or death
\item \textsc{wellbeing} - More general harm to individuals

We can include "psychological harm" or "quality of life" under this category, if we do not specifically identify it elsewhere.
This includes the wellbeing (physical, mental, material, etc.) of people inside or outside of the company (system developers, operators, end users, customers...). We may want to consider morale, usability factors, etc. This might also be a reminder to include "positive user experience" among goals.
\item \textsc{liberty and legal} - Violation of laws, contractual obligations, or certification obligations

This includes encountering legal proceedings that are cumbersome or expensive, even if penalties are not incurred.
\item \textsc{lucre} - Financial loss (e.g. Through fines, COGS, inability to sell, ...)
\item \textsc{material} - Damage to equipment and facilities
\item \textsc{interference} -  Damage to or disruption of other systems or processes

This can be useful when considering the impact of one Akamai system on another.
More broadly, we do not want Akamai's operation to damage the internet at large.
"Don't make things worse."
\item \textsc{reputation} - Damage to business reputation in a way that erodes customer, employee, or public trust and may impact current or future relationships
\item \textsc{adoption failure} - Shifting from an old system to a new system doesn't occur, occurs too slowly, or is arduous or fraught

This might be related to a mission goal for a transition project of some sort.
\item \textsc{[past losses]} - This is a reminder to examine past accidents for inspiration about losses to avoid.
\end{enumerate}  
 
\newthought{Strategic Approaches} 

Ask "what is \emph{unacceptable}?" vs. "what is \emph{risky but tolerable}?" to distinguish from hazards.

\newthought{Relationship to other concepts}

For each of these \textbf{losses}, we will identify hazardous system conditions that, in combination with environmental conditions, can result in an accident in which we experience the loss.

The relationship between \textbf{losses} and \textbf{hazards} lets us \emph{prioritize} our safety efforts, focusing on preventing the system states that are relevant to producing these accidents--- we don't need to examine every combination of system states.
 
\newtheorem{example}{Ex}

\setlength{\parindent}{0em}
\setlength{\parskip}{.75em}

\section{Identifying Hazards}

\newthought{Definition}

\textbf{Hazards} are system conditions that, in combination with environmental conditions outside our control, can result in a loss.

Our task is to \textbf{write a list} of hazards, staying broadly general and covering all the losses.

For each of the \textbf{losses} defined earlier, we'll identify one or more \textbf{hazards}. 



\newthought{Thermostat Example}

\input{examples/thermostat/losses_table}
\input{examples/thermostat/hazards}

\newthought{Desired Qualities} 
\begin{itemize}
\setlength{\itemsep}{0pt}
\setlength{\parskip}{.25em}
\item Concise --- We want a relatively short list
\item General --- We don't want to prematurely narrow our focus. 
\item Good coverage --- For any accident we can think up, we want it to be described by at least one of the hazards on this list.
\item Non-redundant --- Overlap between hazards is ok, but if one loss is entirely a subset of another, perhaps consider consolidating them.
\item Under our control --- For them to be useful in guiding our actions, hazards should identify conditions we can actually do something about. Things outside our control (like weather, meteors, or the popularity of particular websites) are environmental conditions.
\item Relevant --- They should be associated with the losses in a meaningful way. 

Perhaps list what environmental condition would result in the loss.
\end{itemize}  
 
\newthought{Strategic Approaches} 

Ask "what is \emph{risky but tolerable}?" vs. "what is \emph{unacceptable}?" to distinguish from losses--- What is a priority?

\newthought{Relationship to other concepts}

The relationship between \textbf{losses} and \textbf{hazards} lets us \emph{prioritize} our safety efforts, focusing on preventing the system states that are relevant to producing these accidents--- we don't need to examine every combination of system states.

\section{Identifying Unsafe Control Actions}

\newthought{Definition}

Now we ask, "How could the system enter hazardous states?"

We focus on how actions taken by parts of the system could cause hazards, if those actions occur under inappropriate conditions. 
We use a table with these guides:\begin{itemize}
\setlength{\itemsep}{0pt}
\setlength{\parskip}{.25em}
\item Control action required for safety is not provided or not followed
\item An unsafe control action is provided that leads to a hazard (action inappropriate)
\item A potentially safe control action is provided too late, too early, or out of sequence (occurs at wrong time)
\item A safe control action is stopped too soon or applied too long (occurs for wrong duration)
\end{itemize}

Each cell entry is marked with ``not applicable'', ``not hazardous'', or a description of the \textbf{context} that makes the action dangerous and the \textbf{hazard} that results. 

\newthought{Thermostat Example}

\input{examples/thermostat/unsafe_control_actions}

\newthought{Desired Qualities} 
\begin{itemize}
\setlength{\itemsep}{0pt}
\setlength{\parskip}{.25em}
\item Completeness --- Check each cell.
\item Each cell entry is marked with ``not applicable'', ``not hazardous'', or a description of the \textbf{context} that makes the action dangerous and the \textbf{hazard} that results. 
\item It is acceptable for a cell to include multiple (context,hazard) pairs.
\end{itemize}

\newthought{Strategic Approaches} 

\begin{itemize}
\setlength{\itemsep}{0pt}
\setlength{\parskip}{.25em}
\item \textsc{Divide and conquer} --- This work can be split up among several people.
\item \textsc{Pace your efforts} --- Track progress, returning later as needed.
\item Triage and prioritize --- This table can get huge. With limited time resources, we may choose not to aim for completeness.

Prioritize investigating actions needing additional scrutiny, e.g. actions that are new, poorly understood, currently changing, believed to be risky, or involved in close interactions with other systems that need additional scrutiny.
\end{itemize}  

\section{Unsafe Control Actions Table}

\newthought{Requirements}

For this step, you will need:
\begin{itemize}
\item A list of \textbf{actions} to analyze (taken from your \textbf{control model})
\item A list of \textbf{hazards} those actions might cause if misapplied somehow
\end{itemize}

In each cell, consider the action at the start of the row, and what hazard or hazards might result if that action occurs in the way described at the top of the column. 

In that cell, write down the hazard number and the context or condition under which that application of the action would result in that hazard.

\begin{enumerate}
\item Control action \hbox{required} for safety is not provided or not followed. (not provided)
\item An unsafe control action is provided that leads to a hazard.  (provided but wrong)
\item A potentially safe control action is provided too late, too early, or out of sequence.
\item A safe control action is stopped too soon or applied too long. (applied for wrong duration)
\end{enumerate}

\begin{table}
\renewcommand{\arraystretch}{3}
\begin{tabular}{|p{3cm}|p{3cm}|p{3cm}|p{3cm}|p{3cm}|}
\hline
\textsc{Control action}&\textbf{not provided}&\textbf{provided but wrong}&\textbf{too late, too early, or out of sequence}&\textbf{applied for wrong duration}\\
\hline
&&&&\\
\hline
&&&&\\
\hline
&&&&\\
\hline
&&&&\\
\hline
&&&&\\
\hline
&&&&\\
\hline
&&&&\\
\hline
\end{tabular}
\vspace{1em}
\end{table}

\newthought{Relationship to other concepts}

The \textbf{actions} we are analyzing here come from our \textbf{control model}.

The judgement of whether an action occurring inappropriately (or inappropriately failing to occur) is a \emph{problem} comes from associating it with a \textbf{hazard}, which we have identified as a precursor to at least one \textbf{loss}.

\section{Identifying Causal Scenarios}

\newthought{Definition}

Now we ask, ?What could cause an operator or part of the system to take an inappropriate action or fail to take action when needed?? A \textbf{causal scenario} is a description of how and why the \textbf{unsafe control action} could happen.

For each \textbf{unsafe control action}, we will inspect the \textbf{control loop} that action is part of in the \textbf{hierarchical control model}, indentifying how flaws in that control loop could cause that action to occur in that particular inappropriate way.

Consider each of the control loop flaws, or \textbf{causal factors}, marked on the control loop diagram, and ask ?How could that factor contribute to this unsafe control action?? 

Once we have a list of causal scenarios, we can check whether our system has mechanisms to address them.

\newthought{Causal Factors Prompt}

[Based on a diagram from\emph{Engineering A Safer World} p.223]
\begin{center}
\includegraphics[width=9cm]{generic_control_loop-flaws}
\end{center}

\newthought{Desired Qualities}

\begin{compactitem}
\item Completeness --- When finding causal scenarios for a particular unsafe control action, consider each part of the control loop in which the action occurs.
\item Plausibility --- Could this happen at all?
\item Relevance
\end{compactitem}

\newthought{Strategic Approaches} 

This is an exercise in focused brainstorming; we might not think of everything, but focusing on the system bit by bit may make it easier to come up with potential problems we hadn?t considered earlier.

\begin{compactitem}
\item \textsc{Divide and conquer} --- This work can be split up among several people.
\item \textsc{Pace your efforts} --- Track progress, returning later as needed.
\item \textsc{Triage and prioritize} --- Prioritize investigating actions needing additional scrutiny, e.g. actions that are new, poorly understood, currently changing, be-lieved to be risky, or involved in close interactions with other systems that need additional scrutiny.
\end{compactitem}  

\section{Control Loop With Causal Factors}

This diagram shows common flaws in a control loop, or \textbf{causal factors}, that might contribute to the controller misapplying an action or failing to take an action it should have, which we identify as an \textbf{unsafe control action} if it results in a \textbf{hazard}. We use this diagram to guide our brainstorming about why each \textbf{unsafe control action} might occur, identifying \textbf{causal scenarios} we can design features to prevent.

\begin{center}
\includegraphics[width=4in]{../images/generic_control_loop-flaws.png}
\end{center}

[Based on a diagram from \emph{Engineering A Safer World} p.223]

\newthought{Relationship to other concepts}

Each \textbf{unsafe control action} maps to several \textbf{causal scenarios}.

Each \textbf{causal scenario} maps to at least one unsafe control action.

When performing this step, we use the \textbf{hierarchical control model} to identify system parts relevant to the action being inspected.

When we check our list of causal scenarios to see whether our system has mechanisms to address them, we identify new system requirements.

\pagebreak

 \newthought{Thermostat Example}
 
\input{examples/thermostat/causal_scenarios}

\end{document}
